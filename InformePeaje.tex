\documentclass[12pt]{article}
\usepackage[spanish]{babel}
\usepackage[utf8]{inputenc}
\usepackage{graphicx}
\usepackage{amsmath}
\usepackage{geometry}
\geometry{a4paper, margin=2.5cm}
\title{Simulación de Eventos Discretos: Sistema de Peaje con Múltiples Cabinas}
\author{Melani Forsythe Matos}
\date{}

\begin{document}
\maketitle

\section*{S1. Introducción}
Este proyecto tiene como objetivo simular un sistema de peaje con múltiples cabinas utilizando un modelo de eventos discretos. Se busca analizar cómo el número de cabinas y la tasa de llegada de vehículos afectan el tiempo de espera promedio de los usuarios.

\section*{S2. Detalles de Implementación}
La simulación se ha implementado en Python, utilizando una estructura basada en eventos con una cola de prioridad (heap). Se modelan dos tipos de eventos: llegada de un vehículo y salida de un vehículo atendido.

Los parámetros básicos de la simulación fueron:
\begin{itemize}
    \item Tasa de llegada: $\lambda = 1/5$ (un vehículo cada 5 segundos, en promedio)
    \item Tasa de servicio: $\mu = 1/4$ (una atención cada 4 segundos, en promedio)
    \item Tiempo total de simulación: 25, 50 y 100 segundos
    \item Número de cabinas: 3 y 4 (dependiendo del experimento)
\end{itemize}

\section*{S3. Resultados y Experimentos}

\subsubsection*{Simulación 1 — 3 cabinas, 50 segundos}
\begin{itemize}
    \item Tiempo promedio de espera: \textbf{0.00 segundos}
    \item Número total de vehículos atendidos: \textbf{11}
    \item Porcentaje de ocupación por cabina:
    \begin{itemize}
        \item Cabina 1: 103.09\%
        \item Cabina 2: 40.04\%
        \item Cabina 3: 14.37\%
    \end{itemize}
\end{itemize}

\subsubsection*{Simulación 2 — 3 cabinas, 100 segundos}
\begin{itemize}
    \item Tiempo promedio de espera: \textbf{0.06 segundos}
    \item Número total de vehículos atendidos: \textbf{20}
    \item Porcentaje de ocupación por cabina:
    \begin{itemize}
        \item Cabina 1: 149.52\%
        \item Cabina 2: 39.64\%
        \item Cabina 3: 28.66\%
    \end{itemize}
\end{itemize}

\subsubsection*{Simulación 3 — 4 cabinas, 500 segundos}
\begin{itemize}
    \item Tiempo promedio de espera: \textbf{0.04 segundos}
    \item Número total de vehículos atendidos: \textbf{102}
    \item Porcentaje de ocupación por cabina:
    \begin{itemize}
        \item Cabina 1: 873.72\%
        \item Cabina 2: 470.03\%
        \item Cabina 3: 180.06\%
        \item Cabina 4: 51.38\%
    \end{itemize}
\end{itemize}
Para validar el modelo, se realizaron varias ejecuciones y se observó una tendencia coherente entre el número de cabinas disponibles y la disminución del tiempo de espera.

\subsection*{Hipótesis}
Si aumentamos el número de cabinas, el tiempo promedio de espera disminuye. También, si aumentamos la tasa de llegada, el sistema se congestiona más rápidamente.

\section*{S4. Modelo Matemático (Ampliado)}

El sistema puede modelarse utilizando un modelo de colas $M/M/c$, donde:

\begin{itemize}
    \item Las llegadas siguen un proceso de Poisson con tasa $\lambda$
    \item Los tiempos de servicio siguen una distribución exponencial con media $1/\mu$
    \item Hay $c$ servidores (cabinas)
\end{itemize}

La probabilidad de que un cliente tenga que esperar, conocida como fórmula de Erlang-C, está dada por:

\[
P_{espera} = \frac{ \frac{ (\lambda/\mu)^c }{c!} \cdot \frac{c\mu}{c\mu - \lambda} }{ \sum_{n=0}^{c-1} \frac{ (\lambda/\mu)^n }{n!} + \frac{ (\lambda/\mu)^c }{c!} \cdot \frac{c\mu}{c\mu - \lambda} }
\]

El tiempo promedio de espera en cola está dado por:

\[
W_q = \frac{P_{espera}}{c\mu - \lambda}
\]

Y el tiempo promedio total en el sistema es:

\[
W = W_q + \frac{1}{\mu}
\]

Este modelo teórico nos permite comparar los resultados simulados con una base matemática. En este proyecto, sin embargo, se opta por una simulación debido a la flexibilidad de los parámetros.

\section*{S5. Conclusiones}
La simulación demuestra que el número de cabinas impacta directamente en el rendimiento del sistema de peaje. A medida que se agregan cabinas, el tiempo promedio de espera disminuye, especialmente en escenarios de alta demanda. Este modelo puede extenderse fácilmente para estudiar otros escenarios como tiempos de servicio no exponenciales, cabinas prioritarias o llegada en oleadas.

\end{document}


